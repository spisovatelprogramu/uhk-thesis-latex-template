\section{Jak pracovat s touto šablonou}
\label{jak}

\begin{Large}
  \begin{itemize}
    \item Veškerý text píšete do souboru \texttt{text-01-kapitoly.tex}.
    \item Obsah text-02-literatura.bib nahraďte integrací zotera.
  \end{itemize}
\end{Large}

Tento text je převzat z VUT šablony.
\todo{Nahradit vlastními přílohami}

Nezapomeňte, že vlna neřeší všechny nezlomitelné mezery. Vždy je třeba manuální kontrola, zda na konci řádku nezůstalo něco nevhodného -- viz Internetová jazyková příručka\footnote{Internetová jazyková příručka \url{http://prirucka.ujc.cas.cz/?id=880}}.

V této příloze je uveden popis jednotlivých částí šablony, po kterém následuje stručný návod, jak s touto šablonou pracovat. Pokud po jejím přečtení k šabloně budete mít nějaké dotazy, připomínky apod., neváhejte a napište na e-mail \texttt{dobromi1@uhk.cz}.


\section*{Doporučený postup práce se šablonou}

Obsah práce se generuje standardním příkazem \tt \textbackslash tableofcontents \rm (zahrnut v šabloně). Přílohy jsou v něm uvedeny úmyslně.

\subsection*{Styl odstavců}

Odstavce se zarovnávají do bloku a pro jejich formátování existuje více metod. U papírové literatury je častá metoda s~použitím odstavcové zarážky, kdy se u~jednotlivých odstavců textu odsazuje první řádek odstavce asi o~jeden až dva čtverčíky, tedy přibližně o~dvě šířky velkého písmene M základního textu (vždy o~stejnou, předem zvolenou hodnotu). Poslední řádek předchozího odstavce a~první řádek následujícího odstavce se v~takovém případě neoddělují svislou mezerou. Proklad mezi těmito řádky je stejný jako proklad mezi řádky uvnitř odstavce.

Další metodou je odsazení odstavců, které je časté u elektronické sazby textů. První řádek odstavce se při této metodě neodsazuje a mezi odstavce se vkládá vertikální mezera o~velikosti 1/2 řádku. Obě metody lze v kvalifikační práci použít, nicméně často je vhodnější druhá z uvedených metod. Metody není vhodné kombinovat.

Jeden z výše uvedených způsobů je v šabloně nastaven jako výchozí, druhý můžete zvolit parametrem šablony \uv{\tt odsaz\rm }.

\subsection*{Užitečné nástroje}
\label{nastroje}

Následující seznam není výčtem všech využitelných nástrojů. Máte-li vyzkoušený osvědčený nástroj, neváhejte jej využít. Pokud však nevíte, který nástroj si zvolit, můžete zvážit některý z následujících:


\section{Itemize} 
\label{itemize}

\footnote{\url{http://blog.igor.szoke.cz/2017/04/predstartovni-priprava-letu-neni.html}}. 

Velká bezpečnost letecké dopravy stojí z části na tom, že lidé kolem letadel mají \textbf{itemizey} na úplně každý, třeba rutinní a dobře zažitý, postup. Jako pilot strpí to, že bude trochu za blbce a opravdu tužtičkou do seznamu úkonů odškrtá dokonale zvládnuté akce, vytiskněte si a odškrtejte před odevzdáním diplomky i vy tento itemize a vyhněte se tak častým chybám, které by mohly mít až fatální následky na výsledné hodnocení Vaší práce.

\subsubsection*{Struktura}
\begin{itemize}
	\item Už ze samotných názvů a struktury kapitol je patrné, že bylo splněno zadání.
	\item V textu se nevyskytuje kapitola, která by měla méně než čtyři strany (kromě úvodu a závěru). Pokud ano, radil(a) jsem se o tom s vedoucím a ten to schválil.
\end{itemize}

\subsubsection*{Obrázky a grafy}
\begin{itemize}
	\item Všechny obrázky a tabulky byly zkontrolovány a jsou poblíž místa, odkud jsou z textu odkazovány, takže nebude problém je najít.
	\item Všechny obrázky a tabulky mají takový popisek, že celý obrázek dává smysl sám o~sobě, bez čtení dalšího textu. Vůbec nevadí, když má popisek několik řádků.
	\item Pokud je obrázek převzatý, tak je to v popisku zmíněno: \uv{Převzato z [X].}
	\item Písmenka ve všech obrázcích používají font podobné velikosti, jako je okolní text (ani výrazně větší, ani výrazně menší).
	\item Grafy a schémata jsou vektorově (tj. v PDF).
	\item Snímky obrazovky nepoužívají ztrátovou kompresi (jsou v PNG).
	\item Všechny obrázky jsou odkázány z textu.
	\item Grafy mají popsané osy (název osy, jednotky, hodnoty) a podle potřeby mřížku.
\end{itemize}

\subsubsection*{Rovnice}
\begin{itemize}
	\item Identifikátory a jejich indexy v rovnicích jsou jednopísmenné (kromě nečastých zvláštních případů jako $t_\mathrm{max}$).
	\item Rovnice jsou číslovány.
	\item Za (nebo vzácně před) rovnicí jsou vysvětleny všechny proměnné a funkce, které zatím vysvětleny nebyly.
\end{itemize}

\subsubsection*{Citace}
\begin{itemize}
    \item \textbf{Všechny použité zdroje jsou citovány.}
	\item Adresy URL odkazující na služby, projekty, zdroje, github apod. jsou odkazovány pomocí \verb|\footnote{\url{...}}|.
    \item Všechny citace používají správné typy.
	\item Citace mají autora, název, vydavatele (název konference), rok vydání.  Když některá nemá, je to dobře zdůvodněný zvláštní případ a vedoucí to odsouhlasil.
	\item Je-li ve zdrojových textech programu něco převzaté, je to tam řádně citováno v souladu s licencí.
	\item Je-li podstatná část zdrojových textů programu převzatá, je toto zmíněno v textu práce a je citován zdroj.
\end{itemize}

\subsubsection*{Typografie}
\begin{itemize}
	\item Žádný řádek nepřetéká přes pravý okraj.
	\item Na konci řádku nikde není jednopísmenná předložka (spraví to nedělitelná mezera $\sim$).
	\item Číslo obrázku, tabulky, rovnice, citace není nikde první na novém řádku (spraví to nedělitelná mezera $\sim$).
	\item Před číselným odkazem na poznámku pod čarou nikde není mezera (to jest vždy takto\footnote{příklad poznámky pod čarou}, nikoliv takto \footnote{jiný příklad poznámky pod čarou}).
\end{itemize}

\subsubsection*{Jazyk}
\begin{itemize}
    \item Použil jsem kontrolu pravopisu a v textu nikde nejsou překlepy.
	\item Nechal jsem si text přečíst od (alespoň) jednoho dalšího člověka, který umí dobře česky / anglicky / slovensky.
	\item V práci psané česky nebo slovensky abstrakt zkontroloval někdo, kdo umí opravdu dobře anglicky.
	\item V textu se nikde nepoužívá druhá mluvnická osoba (vy/ty).
	\item Když se v textu vyskytuje první mluvnická osoba (já, my), vždy se popisuje subjektivní záležitost (\textit{rozhodl jsem se}, \textit{navrhl jsem}, \textit{zaměřil jsem se na}, \textit{zjistil jsem} apod.).
	\item V textu se nikde nepoužívají hovorové výrazy.
	\item V českém či slovenském textu se zbytečně nepoužívají anglické výrazy, které mají ustálené české překlady. Např. slovo \textit{defaultní} se nahradí např. slovem \textit{implicitní} nebo \textit{výchozí}.
\end{itemize}

\subsubsection*{Výsledek na datovém médiu, tj. software}
\begin{itemize}
	\item Mám připravené nepřepisovatelné datové médium 
      \begin{itemize}
	  		\item CD-R,
            \item DVD-R,
            \item DVD+R ve formátu ISO9660 (s rozšířením RockRidge a/nebo Jolliet) nebo UDF,
            \item paměťová karta SD (Secure Digital) ve formátu FAT32 nebo exFAT s nastavenou ochranou proti přepisu.
      \end{itemize}
	\item Pokud je výsledek online (služba, aplikace, \dots), URL je viditelně v úvodu a závěru, aby bylo jasné, kde výsledek hledat.
	\item Na médiu nechybí povinné: 
    	\begin{itemize}
    		\item zdrojové kódy (např. Matlab, C/C++, Python, \dots)
            \item knihovny potřebné pro překlad,
            \item přeložené řešení,
            \item PDF s technickou zprávou (je-li pro tisk 2. verze, tak obě),
            \item zdrojový kód zprávy (\LaTeX), 
    	\end{itemize}
        a případně volitelně po dohodě s vedoucím práce
		\begin{itemize}
			\item relevantní (např. testovací) data, 
            \item demonstrační video,
            \item PDF plakátku,
            \item \dots
		\end{itemize}        
	\item Zdrojové kódy jsou refaktorovány, komentovány a označeny hlavičkou s autorstvím, takže se v nich snadno vyzná i někdo další, než sám autor.
    \item Jakákoliv převzatá část zdrojového kódu je řádně citována -- tedy označena úvodním a v případě převzetí více řádků i ukončovacím komentářem. Komentář obsahuje vše, co vyžaduje licence uvedená na webu (vždy je nutné se ji pokusit najít -- např. Stack Overflow\footnote{\url{https://stackoverflow.blog/2009/06/25/attribution-required/}} má striktní pravidla pro citace).
\end{itemize}

\subsubsection*{Odevzdání}

\begin{itemize}
\item Pokud chci tisknout oboustranně, konzultoval(a) jsem to s~vedoucím a mám správně nastavenou šablonu. Kapitoly začínají na liché stránce.
\item Technickou zprávu mám v deskách z knihařství (min. 1 výtisk, při utajení oba).
\item Za titulním listem práce je zadání (tzn. mám jej stažené z IS a vložené do šablony).
\item V IS jsou abstrakty a klíčová slova.
  \begin{itemize}
    \item V abstraktu a klíčových slovech v IS nejsou zkopírované vlnky pro nezlomitelné mezery.
  \end{itemize}      
\item V IS je PDF práce (s klikatelnými odkazy).
\item Oba výtisky práce jsou podepsané.
\item V jednom (při utajení obou) výtisku práce je paměťové médium, na kterém je fixkou napsaný login (fixku na CD lze zapůjčit v knihovně, na Studijním oddělení nebo až při odevzdání).
\end{itemize}


\section{\LaTeX pro začátečníky}
\label{latex}

V této kapitole jsou uvedeny některé často využívané balíčky a příkazy pro \LaTeX{}, které mohou být při tvorbě práce potřeba.

\subsection*{Užitečné balíčky}

Studenti při sazbě textu často řeší stejné problémy. Některé z nich lze vyřešit následujícími balíčky pro \LaTeX:

\begin{itemize}
  \item \verb|amsmath| -- rozšířené možnosti sazby rovnic,
  \item \verb|float, afterpage, placeins| -- úprava umístění obrázků/tabulek (specifikátor \texttt{H}),
  \item \verb|fancyvrb, alltt| -- úpravy vlastností prostředí Verbatim, 
  \item \verb|makecell| -- rozšíření možností tabulek,
  \item \verb|pdflscape, rotating| -- natočení stránky o 90 stupňů (pro obrázek či tabulku),
  \item \verb|hyphenat| -- úpravy dělení slov,
  \item \verb|picture, epic, eepic| -- přímé kreslení obrázků.
\end{itemize}

Některé balíčky jsou využity přímo v šabloně (v dolní části souboru \texttt{fitthesis.cls}). Nahlédnutí do jejich dokumentace může být rovněž velmi užitečné.

Sloupec tabulky zarovnaný vlevo s pevnou šířkou je v šabloně definovaný \uv{L} (používá se jako \uv{p}).

Pro odkazování v rámci textu použijte příkaz \verb|\ref{navesti}|. Podle umístění návěští se bude jednat o~číslo kapitoly, podkapitoly, obrázku, tabulky nebo podobného číslovaného prvku). Pokud chcete odkázat stránku práce, použijte příkaz \verb|pageref{navesti}|. Pro citaci literárního odkazu \verb|\cite{identifikator}|. Pro odkazy na rovnice lze použít příkaz \verb|\eqref{navesti}|.

Znak \,--\, (pomlčka) se V \LaTeX u vkládá jako dvě mínus za sebou: -{}-.

\subsection*{Často využívané příkazy pro \LaTeX{}}
\label{sec:Fragments}

Doporučuji nahlédnout do zdrojového textu této podkapitoly a podívat se, jak jsou následující ukázky vysázeny. Ve zdrojovém textu jsou i pomocné komentáře.

% Sloupec zarovnaný vlevo s pevnou šířkou je v šabloně definovaný "L" (používá se jako p)

Příklad tabulky:
\begin{table}[H]
	\vskip6pt
	\caption{Tabulka hodnocení} 
    \vskip6pt
	\centering
	\begin{tabular}{llr}
		\toprule
		\multicolumn{2}{c}{Jméno} \\
		\cmidrule(r){1-2}
		Jméno & Příjmení & Hodnocení \\
		\midrule
		Jan & Novák & $7.5$ \\
		Petr & Novák & $2$ \\
		\bottomrule
	\end{tabular}
	\label{tab:ExampleTable}
\end{table}

% Ohraničení lze upravit dle potřeby:
% http://latex-community.org/forum/viewtopic.php?f=45&t=24323
% http://tex.stackexchange.com/questions/58163/problem-with-multirow-and-table-cell-borders
% http://tex.stackexchange.com/questions/79369/formatting-table-border-and-text-alignment-in-latex-table

\noindent Příklad rovnice:
\begin{equation}
	\cos^3 \theta =\frac{1}{4}\cos\theta+\frac{3}{4}\cos 3\theta
	\label{eq:rovnice2}
\end{equation}
a dvou horizontálně zarovnaných rovnic: % znak & řídí zarovnání
\begin{align} 
    \label{eq:soustava}
	3x &= 6y + 12 \\
	x &= 2y + 4 
\end{align}

Pokud je třeba rovnici citovat v textu, lze použít příkaz \verb|\eqref|. Například na rovnici výše lze odkázat~\eqref{eq:rovnice2}. Pokud chcete srovnat číslo rovnic u soustavy, lze použít prostředí \texttt{split}:
\begin{equation} \label{eq:soustavaSrovnana}
\begin{split}
	3x &= 6y + 12 \\
	x &= 2y + 4
\end{split}
\end{equation}

Matematické symboly ($\alpha$) a výrazy lze umístit i do textu $\cos\pi=-1$ a mohou být i~v~poznámce pod čarou%
\footnote{Vzorec v poznámce pod čarou: $\cos\pi=-1$}.

Obrázek~\ref{sirokyObrazek} ukazuje široký obrázek složený z více menších obrázků. Klasický rastrový obrázek se vkládá tak, jak je vidět na obrázku \ref{keepCalm}.

% Využití \begin{figure*} způsobí, že obrázek zabere celou šířku stránky. Takový obrázek dříve mohl být pouze na začátku stránky, případně na konci s využitím balíčku dblfloatfix (případné [h] se ignorovalo a [H] obrázek odstraní). Nové verze LaTeXu už umí i [h].
\begin{figure*}[h]\centering
  \centering
  \includegraphics[width=\linewidth,height=1.7in]{obrazky/uhk.jpg}\\[1pt]
  \includegraphics[width=0.24\linewidth]{obrazky/uhk.jpg}\hfill
  \includegraphics[width=0.24\linewidth]{obrazky/uhk.jpg}\hfill
  \includegraphics[width=0.24\linewidth]{obrazky/uhk.jpg}\hfill
  \includegraphics[width=0.24\linewidth]{obrazky/uhk.jpg}
  \caption{\textbf{Široký obrázek.} Obrázek může být složen z více menších obrázků. Chcete-li se na tyto dílčí obrázky odkazovat z textu, využijte balíček \texttt{subcaption}.}
  \label{sirokyObrazek}
\end{figure*}

% Odkomentujte pro přepnutí na formát A3 na šířku
% \eject \pdfpagewidth=420mm

\begin{figure}[hbt]
	\centering
	\includegraphics[width=0.3\textwidth]{obrazky/uhk.jpg}
	\caption{Dobrý text je špatným textem, který byl několikrát přepsán. Nebojte se prostě něčím začít.}
	\label{keepCalm}
\end{figure}

Někdy je potřeba do příloh umístit diagram, který se nevejde na stránku formátu A4. Pak je možné vložit jednu stránku formátu A3 a do práce ji poskládat (tzv. skládání do~Z, kdy se vytvoří dva sklady -- lícem dolů a lícem nahoru, angl. Engineering fold -- existuje i~anglický pojem Z-fold, ale při tom by byl problém s vazbou). Přepnutí se provádí následovně: \texttt{\textbackslash{}eject \textbackslash{}pdfpagewidth=420mm} (pro přepnutí zpět pak 210mm).

Další často využívané příkazy naleznete ve zdrojovém textu ukázkového obsahu této šablony.

% Odkomentujte pro přepnutí zpět na A4
% \eject \pdfpagewidth=210mm

