\documentclass[]{fim-uhk-thesis}
%\documentclass[english]{fim-uhk-thesis} % without assignment - for the work start to avoid compilation problem
% * Je-li práce psaná v anglickém jazyce, je zapotřebí u třídy použít 
%   parametr english následovně:
%   If thesis is written in English, it is necessary to use 
%   parameter english as follows:
%      \documentclass[english]{fim-uhk-thesis}
% * Je-li práce psaná ve slovenském jazyce, je zapotřebí u třídy použít 
%   parametr slovak následovně:
%   If the work is written in the Slovak language, it is necessary 
%   to use parameter slovak as follows:
%      \documentclass[slovak]{fim-uhk-thesis}
% * Je-li práce psaná v anglickém jazyce se slovenským abstraktem apod., 
%   je zapotřebí u třídy použít parametry english a enslovak následovně:
%   If the work is written in English with the Slovak abstract, etc., 
%   it is necessary to use parameters english and enslovak as follows:
%      \documentclass[english,enslovak]{fim-uhk-thesis}

% Základní balíčky jsou dole v souboru šablony fim-uhk-thesis.cls
% Basic packages are at the bottom of template file fim-uhk-thesis.cls
% zde můžeme vložit vlastní balíčky / you can place own packages here



%---rm---------------
\renewcommand{\rmdefault}{lmr}%zavede Latin Modern Roman jako rm / set Latin Modern Roman as rm
%---sf---------------
\renewcommand{\sfdefault}{qhv}%zavede TeX Gyre Heros jako sf
%---tt------------
\renewcommand{\ttdefault}{lmtt}% zavede Latin Modern tt jako tt

% vypne funkci šablony, která automaticky nahrazuje uvozovky,
% aby nebyly prováděny nevhodné náhrady v popisech API apod.
% disables function of the template which replaces quotation marks
% to avoid unnecessary replacements in the API descriptions etc.
\csdoublequotesoff



\usepackage{url}


% =======================================================================
% balíček "hyperref" vytváří klikací odkazy v pdf, pokud tedy použijeme pdflatex
% problém je, že balíček hyperref musí být uveden jako poslední, takže nemůže
% být v šabloně
% "hyperref" package create clickable links in pdf if you are using pdflatex.
% Problem is that this package have to be introduced as the last one so it 
% can not be placed in the template file.

  \usepackage{color}
  \usepackage[unicode,colorlinks,hyperindex,plainpages=false,urlcolor=black,linkcolor=black,citecolor=black]{hyperref}
  \definecolor{links}{rgb}{0,0,0}
  \definecolor{anchors}{rgb}{0,0,0}
  \def\AnchorColor{anchors}
  \def\LinkColor{links}
  \def\pdfBorderAttrs{/Border [0 0 0] } % bez okrajů kolem odkazů / without margins around links
  \pdfcompresslevel=9

% Řešení problému, kdy klikací odkazy na obrázky vedou za obrázek
% This solves the problems with links which leads after the picture
\usepackage[all]{hypcap}

\projectinfo{
  %Prace / Thesis
  project={BP},            %typ práce BP/SP/DP/DR  / thesis type (SP = term project)
  year={2020},             % rok odevzdání / year of submission
  date=\today,             % datum odevzdání / submission date
  %Nazev prace / thesis title
  title.cs={Název práce},  % název práce v češtině či slovenštině (dle zadání) / thesis title in czech language (according to assignment)
  title.en={Thesis title}, % název práce v angličtině / thesis title in english
  %title.length={14.5cm}, % nastavení délky bloku s titulkem pro úpravu zalomení řádku (lze definovat zde nebo níže) / setting the length of a block with a thesis title for adjusting a line break (can be defined here or below)
  %sectitle.length={14.5cm}, % nastavení délky bloku s druhým titulkem pro úpravu zalomení řádku (lze definovat zde nebo níže) / setting the length of a block with a second thesis title for adjusting a line break (can be defined here or below)
  %Autor / Author
  author.name={Jméno},   % jméno autora / author name
  author.surname={Příjmení},   % příjmení autora / author surname 
  author.field={Aplikovaná Informatika},   % obor
  %author.title.p={Bc.}, % titul před jménem (nepovinné) / title before the name (optional)
  %author.title.a={Ph.D.}, % titul za jménem (nepovinné) / title after the name (optional)
  %Ustav / Department
  department={DEF}, % doplňte příslušnou zkratku dle ústavu na zadání: DEF / fill in appropriate abbreviation of the department according to assignment: DEF
  % Školitel / supervisor
  supervisor.name={Jméno},   % jméno školitele / supervisor name 
  supervisor.surname={Příjmení},   % příjmení školitele / supervisor surname
  supervisor.title.p={prof. RNDr.},   %titul před jménem (nepovinné) / title before the name (optional)
  supervisor.title.a={Ph.D.},    %titul za jménem (nepovinné) / title after the name (optional)
  % Klíčová slova / keywords
  keywords.cs={Sem budou zapsána jednotlivá klíčová slova v českém (slovenském) jazyce, oddělená čárkami.}, % klíčová slova v českém či slovenském jazyce / keywords in czech or slovak language
  keywords.en={Sem budou zapsána jednotlivá klíčová slova v anglickém jazyce, oddělená čárkami.}, % klíčová slova v anglickém jazyce / keywords in english
  %keywords.en={Here, individual keywords separated by commas will be written in English.},
  % Anotace
  annotation.cs={Text anotace – shrnutí cíle, významu práce a výsledky v ní dosažené. Délka minimálně 100 a maximálně 200 slov},
  annotation.en={Anotace v anglickém jazyce. Délka minimálně 100 a maximálně 200 slov.},
  % Prohlášení (u anglicky psané práce anglicky, u slovensky psané práce slovensky) / Declaration (for thesis in english should be in english)
  declaration={Prohlašuji, že jsem bakalářskou/diplomovou práci zpracoval/zpracovala samostatně a s použitím uvedené literatury},
  %declaration={I hereby declare that this Bachelor's thesis was prepared as an original work by the author under the supervision of Mr. X
% The supplementary information was provided by Mr. Y
% I have listed all the literary sources, publications and other sources, which were used during the preparation of this thesis.},
  % Poděkování (nepovinné, nejlépe v jazyce práce) / Acknowledgement (optional, ideally in the language of the thesis)
  acknowledgment={V této sekci je možno uvést poděkování vedoucímu práce a těm, kteří poskytli odbornou pomoc
(externí zadavatel, konzultant apod.).},
  %acknowledgment={Here it is possible to express thanks to the supervisor and to the people which provided professional help
%(external submitter, consultant, etc.).},
  faculty={FIM}, % TODO FMI/DEF
%   faculty.cs={Fakulta informatiky a managementu}, % Fakulta v češtině - pro využití této položky výše zvolte fakultu DEF / Faculty in Czech - for use of this entry select DEF above
%   faculty.en={Faculty of Informatics and Management}, % Fakulta v angličtině - pro využití této položky výše zvolte fakultu DEF / Faculty in English - for use of this entry select DEF above
  department={KIT} % TODO dopsat seznam
%   department.cs={KATEDRA INFORMAČNÍCH TECHNOLOGIÍ}, % Ústav v češtině - pro využití této položky výše zvolte ústav DEF nebo jej zakomentujte / Department in Czech - for use of this entry select DEF above or comment it out
%   department.en={DEPARTMENT OF INFORMATION TECHNOLOGY} % Ústav v angličtině - pro využití této položky výše zvolte ústav DEF nebo jej zakomentujte / Department in English - for use of this entry select DEF above or comment it out
}

% řeší první/poslední řádek odstavce na předchozí/následující stránce
% solves first/last row of the paragraph on the previous/next page
\clubpenalty=10000
\widowpenalty=10000

% @SEE http://ftp.cvut.cz/tex-archive/macros/latex/contrib/acro/acro-manual.pdf

\DeclareAcronym{DNN}{
  short=DNN,
  long=deep neuron network
}


% začátek dokumentu
\begin{document}
  % Vysazeni titulnich stran / Typesetting of the title pages
  % ----------------------------------------------
  \maketitle
  % Obsah
  % ----------------------------------------------
  \setlength{\parskip}{0pt}

  \thispagestyle{empty}
  {\hypersetup{hidelinks}\tableofcontents}
  
  % Seznam obrazku a tabulek (pokud prace obsahuje velke mnozstvi obrazku, tak se to hodi)
  % List of figures and list of tables (if the thesis contains a lot of pictures, it is good)
  
  \ifczech
    \renewcommand\listfigurename{Seznam obrázků}
  \fi
  \ifslovak
    \renewcommand\listfigurename{Zoznam obrázkov}
  \fi
   {\hypersetup{hidelinks}\listoffigures}
   \thispagestyle{empty}
  
  \ifczech
    \renewcommand\listtablename{Seznam tabulek}
  \fi
  \ifslovak
    \renewcommand\listtablename{Zoznam tabuliek}
  \fi
  
  {\hypersetup{hidelinks}\listoftables}
  \thispagestyle{empty}

  % vynechani stranky v oboustrannem rezimu
  % Skip the page in the two-sided mode
  \iftwoside
    \cleardoublepage
  \fi

  % Text prace / Thesis text
  % ----------------------------------------------
  \setcounter{page}{0}
  % ################################
\section{Úvod}

\todo{Zde vysvětlit problémovou situaci a otázky, které se budou v bakalářské/diplomové práci řešit.}

Citace \cite{abadi_tensorflow:_2016, lacey_deep_2016}.


% ################################
\section{Cíl práce}

\todo{Smysl a účel, výzkumné otázky.}


% ################################
\section{Metodika zpracování}

\todo{Cíle, hypotézy/ výzkumné otázky, způsob hledání odpovědí na výzkumné otázky včetně metodiky vlastního výzkumu/šetření, literární rešerše.}


% ################################
\section{Vlastní text práce}

\todo{TODO}

Vlastní řešení dokládá student zpravidla v několika kapitolách. Podle charakteru práce musí student uvážit, zda informace netextové povahy (data, tabulky, obrázky atd.) bude uvádět přímo v textu, nebo je zařadí až za celou práci ve formě příloh, či bude kombinovat oba způsoby. 

Více podrobností viz Metodické pokyny pro vypracování bakalářských a diplomových prací (zveřejňované formou výnosů děkana) a v kurzu MES – Metodologický seminář. 

\subsection{Podkapitola}

Vlastní text práce.

\begin{table}[hbt!]  
\caption{průměry}
\centering
\begin{tabular}{| l | r | r | r | }
\hline
        &        psnr &      ssim &      doba  \\
model &       (db)    &           & gen. (s) \\
\hline
bik. int. & 28.3155 & 0.8566 & 0.0322 \\
nn1000    & 30.1461 & 0.9043 & 0.8109 \\
nn1001    & 30.0324 & 0.9023 & 0.7486 \\
nn1002    & \textbf{30.1886} & \textbf{0.9046} & 1.1731 \\
nn1003    & 30.0390 & 0.9030 & 1.1320 \\
nn1004    & 24.9772 & 0.7172 & 4.4367 \\
nn1005    & 26.1629 & 0.8004 & 4.0475 \\
nn1006    & 27.9129 & 0.8438 & 4.0683 \\
nn1007    & 27.5834 & 0.8360 & 4.2082 \\
\hline
\end{tabular}
\end{table}


\subsubsection{Podřazená kapitola}
Text s odkazem na obrázek \ref{figure:uhk}.


\begin{figure}[hbt!]
 	\begin{center}
    	\includegraphics[width=\textwidth]{obrazky/uhk.jpg}
 	\end{center}
 	\caption{Ukázkový obrázek}
	\label{figure:uhk}
\end{figure} 


% ################################
\section{Souhrn výsledků}

\todo{Souhrn vlastních výsledků získaných v průběhu řešení problému.}

\noindent\todo{Zkratka} \Ac{DNN}


% ################################
\section{Závěry a doporučení}

\todo{Kritická diskuze nad výsledky, ke kterým autor dospěl (soulad výsledků  literaturou či předpoklady; výsledky a okolnosti, které zvláště ovlivnily předkládanou práci atd.). Je vhodné naznačit i případné další (popř. alternativní) možnosti zkoumání dané problematiky a otevřené problémy pro další studium. }


  \iftwoside
    \cleardoublepage
  \fi

  % Pouzita literatura / Bibliography
  % ----------------------------------------------

\ifslovak
  \makeatletter
  \def\@openbib@code{\addcontentsline{toc}{section}{Literatúra}}
\else
  \ifczech
    \makeatletter
    \def\@openbib@code{\addcontentsline{toc}{section}{Literatura}}
  \else 
    \makeatletter
    \def\@openbib@code{\addcontentsline{toc}{section}{Bibliography}}
  \fi
\fi
  \makeatother
  \begin{flushleft}
  \label{references}

  \bibliographystyle{iso690}
  \bibliography{text-02-literatura}
  \end{flushleft}
  
  \newpage

  % vynechani stranky v oboustrannem rezimu
  % Skip the page in the two-sided mode
  \iftwoside
    \cleardoublepage
  \fi
  
  \ifslovak
   \addcontentsline{toc}{section}{Zoznam zkratek}
  \else
    \ifczech
      \addcontentsline{toc}{section}{Seznam zkratek}
    \else
      \addcontentsline{toc}{section}{List of Abbreviations}
    \fi
  \fi
  \startcontents[section]
  \label{abb}
  \setlength{\parskip}{0pt} 
  % seznam příloh / list of appendices
  % \printcontents[section]{l}{0}{\setcounter{tocdepth}{2}}
  
  \printacronyms[name=Seznam zkratek,heading=section*]
  
  \newpage
  % vynechani stranky v oboustrannem rezimu
  \iftwoside
    \cleardoublepage
  \fi

  % Prilohy / Appendices
  % ---------------------------------------------
  \appendix

  \ifczech
    \renewcommand{\appendixpagename}{Přílohy}
    \renewcommand{\appendixtocname}{Přílohy}
    \renewcommand{\appendixname}{Příloha}
  \fi
  \ifslovak
    \renewcommand{\appendixpagename}{Prílohy}
    \renewcommand{\appendixtocname}{Prílohy}
    \renewcommand{\appendixname}{Príloha}
  \fi
 \appendixpage
  
  % Přílohy / Appendices
  \section{Jak pracovat s touto šablonou}
\label{jak}

\begin{Large}
  \begin{itemize}
    \item Veškerý text píšete do souboru \texttt{text-01-kapitoly.tex}.
    \item Obsah text-02-literatura.bib nahraďte integrací zotera.
  \end{itemize}
\end{Large}

Tento text je převzat z VUT šablony.
\todo{Nahradit vlastními přílohami}

Nezapomeňte, že vlna neřeší všechny nezlomitelné mezery. Vždy je třeba manuální kontrola, zda na konci řádku nezůstalo něco nevhodného -- viz Internetová jazyková příručka\footnote{Internetová jazyková příručka \url{http://prirucka.ujc.cas.cz/?id=880}}.

V této příloze je uveden popis jednotlivých částí šablony, po kterém následuje stručný návod, jak s touto šablonou pracovat. Pokud po jejím přečtení k šabloně budete mít nějaké dotazy, připomínky apod., neváhejte a napište na e-mail \texttt{dobromi1@uhk.cz}.


\section*{Doporučený postup práce se šablonou}

Obsah práce se generuje standardním příkazem \tt \textbackslash tableofcontents \rm (zahrnut v šabloně). Přílohy jsou v něm uvedeny úmyslně.

\subsection*{Styl odstavců}

Odstavce se zarovnávají do bloku a pro jejich formátování existuje více metod. U papírové literatury je častá metoda s~použitím odstavcové zarážky, kdy se u~jednotlivých odstavců textu odsazuje první řádek odstavce asi o~jeden až dva čtverčíky, tedy přibližně o~dvě šířky velkého písmene M základního textu (vždy o~stejnou, předem zvolenou hodnotu). Poslední řádek předchozího odstavce a~první řádek následujícího odstavce se v~takovém případě neoddělují svislou mezerou. Proklad mezi těmito řádky je stejný jako proklad mezi řádky uvnitř odstavce.

Další metodou je odsazení odstavců, které je časté u elektronické sazby textů. První řádek odstavce se při této metodě neodsazuje a mezi odstavce se vkládá vertikální mezera o~velikosti 1/2 řádku. Obě metody lze v kvalifikační práci použít, nicméně často je vhodnější druhá z uvedených metod. Metody není vhodné kombinovat.

Jeden z výše uvedených způsobů je v šabloně nastaven jako výchozí, druhý můžete zvolit parametrem šablony \uv{\tt odsaz\rm }.

\subsection*{Užitečné nástroje}
\label{nastroje}

Následující seznam není výčtem všech využitelných nástrojů. Máte-li vyzkoušený osvědčený nástroj, neváhejte jej využít. Pokud však nevíte, který nástroj si zvolit, můžete zvážit některý z následujících:


\section{Itemize} 
\label{itemize}

\footnote{\url{http://blog.igor.szoke.cz/2017/04/predstartovni-priprava-letu-neni.html}}. 

Velká bezpečnost letecké dopravy stojí z části na tom, že lidé kolem letadel mají \textbf{itemizey} na úplně každý, třeba rutinní a dobře zažitý, postup. Jako pilot strpí to, že bude trochu za blbce a opravdu tužtičkou do seznamu úkonů odškrtá dokonale zvládnuté akce, vytiskněte si a odškrtejte před odevzdáním diplomky i vy tento itemize a vyhněte se tak častým chybám, které by mohly mít až fatální následky na výsledné hodnocení Vaší práce.

\subsubsection*{Struktura}
\begin{itemize}
	\item Už ze samotných názvů a struktury kapitol je patrné, že bylo splněno zadání.
	\item V textu se nevyskytuje kapitola, která by měla méně než čtyři strany (kromě úvodu a závěru). Pokud ano, radil(a) jsem se o tom s vedoucím a ten to schválil.
\end{itemize}

\subsubsection*{Obrázky a grafy}
\begin{itemize}
	\item Všechny obrázky a tabulky byly zkontrolovány a jsou poblíž místa, odkud jsou z textu odkazovány, takže nebude problém je najít.
	\item Všechny obrázky a tabulky mají takový popisek, že celý obrázek dává smysl sám o~sobě, bez čtení dalšího textu. Vůbec nevadí, když má popisek několik řádků.
	\item Pokud je obrázek převzatý, tak je to v popisku zmíněno: \uv{Převzato z [X].}
	\item Písmenka ve všech obrázcích používají font podobné velikosti, jako je okolní text (ani výrazně větší, ani výrazně menší).
	\item Grafy a schémata jsou vektorově (tj. v PDF).
	\item Snímky obrazovky nepoužívají ztrátovou kompresi (jsou v PNG).
	\item Všechny obrázky jsou odkázány z textu.
	\item Grafy mají popsané osy (název osy, jednotky, hodnoty) a podle potřeby mřížku.
\end{itemize}

\subsubsection*{Rovnice}
\begin{itemize}
	\item Identifikátory a jejich indexy v rovnicích jsou jednopísmenné (kromě nečastých zvláštních případů jako $t_\mathrm{max}$).
	\item Rovnice jsou číslovány.
	\item Za (nebo vzácně před) rovnicí jsou vysvětleny všechny proměnné a funkce, které zatím vysvětleny nebyly.
\end{itemize}

\subsubsection*{Citace}
\begin{itemize}
    \item \textbf{Všechny použité zdroje jsou citovány.}
	\item Adresy URL odkazující na služby, projekty, zdroje, github apod. jsou odkazovány pomocí \verb|\footnote{\url{...}}|.
    \item Všechny citace používají správné typy.
	\item Citace mají autora, název, vydavatele (název konference), rok vydání.  Když některá nemá, je to dobře zdůvodněný zvláštní případ a vedoucí to odsouhlasil.
	\item Je-li ve zdrojových textech programu něco převzaté, je to tam řádně citováno v souladu s licencí.
	\item Je-li podstatná část zdrojových textů programu převzatá, je toto zmíněno v textu práce a je citován zdroj.
\end{itemize}

\subsubsection*{Typografie}
\begin{itemize}
	\item Žádný řádek nepřetéká přes pravý okraj.
	\item Na konci řádku nikde není jednopísmenná předložka (spraví to nedělitelná mezera $\sim$).
	\item Číslo obrázku, tabulky, rovnice, citace není nikde první na novém řádku (spraví to nedělitelná mezera $\sim$).
	\item Před číselným odkazem na poznámku pod čarou nikde není mezera (to jest vždy takto\footnote{příklad poznámky pod čarou}, nikoliv takto \footnote{jiný příklad poznámky pod čarou}).
\end{itemize}

\subsubsection*{Jazyk}
\begin{itemize}
    \item Použil jsem kontrolu pravopisu a v textu nikde nejsou překlepy.
	\item Nechal jsem si text přečíst od (alespoň) jednoho dalšího člověka, který umí dobře česky / anglicky / slovensky.
	\item V práci psané česky nebo slovensky abstrakt zkontroloval někdo, kdo umí opravdu dobře anglicky.
	\item V textu se nikde nepoužívá druhá mluvnická osoba (vy/ty).
	\item Když se v textu vyskytuje první mluvnická osoba (já, my), vždy se popisuje subjektivní záležitost (\textit{rozhodl jsem se}, \textit{navrhl jsem}, \textit{zaměřil jsem se na}, \textit{zjistil jsem} apod.).
	\item V textu se nikde nepoužívají hovorové výrazy.
	\item V českém či slovenském textu se zbytečně nepoužívají anglické výrazy, které mají ustálené české překlady. Např. slovo \textit{defaultní} se nahradí např. slovem \textit{implicitní} nebo \textit{výchozí}.
\end{itemize}

\subsubsection*{Výsledek na datovém médiu, tj. software}
\begin{itemize}
	\item Mám připravené nepřepisovatelné datové médium 
      \begin{itemize}
	  		\item CD-R,
            \item DVD-R,
            \item DVD+R ve formátu ISO9660 (s rozšířením RockRidge a/nebo Jolliet) nebo UDF,
            \item paměťová karta SD (Secure Digital) ve formátu FAT32 nebo exFAT s nastavenou ochranou proti přepisu.
      \end{itemize}
	\item Pokud je výsledek online (služba, aplikace, \dots), URL je viditelně v úvodu a závěru, aby bylo jasné, kde výsledek hledat.
	\item Na médiu nechybí povinné: 
    	\begin{itemize}
    		\item zdrojové kódy (např. Matlab, C/C++, Python, \dots)
            \item knihovny potřebné pro překlad,
            \item přeložené řešení,
            \item PDF s technickou zprávou (je-li pro tisk 2. verze, tak obě),
            \item zdrojový kód zprávy (\LaTeX), 
    	\end{itemize}
        a případně volitelně po dohodě s vedoucím práce
		\begin{itemize}
			\item relevantní (např. testovací) data, 
            \item demonstrační video,
            \item PDF plakátku,
            \item \dots
		\end{itemize}        
	\item Zdrojové kódy jsou refaktorovány, komentovány a označeny hlavičkou s autorstvím, takže se v nich snadno vyzná i někdo další, než sám autor.
    \item Jakákoliv převzatá část zdrojového kódu je řádně citována -- tedy označena úvodním a v případě převzetí více řádků i ukončovacím komentářem. Komentář obsahuje vše, co vyžaduje licence uvedená na webu (vždy je nutné se ji pokusit najít -- např. Stack Overflow\footnote{\url{https://stackoverflow.blog/2009/06/25/attribution-required/}} má striktní pravidla pro citace).
\end{itemize}

\subsubsection*{Odevzdání}

\begin{itemize}
\item Pokud chci tisknout oboustranně, konzultoval(a) jsem to s~vedoucím a mám správně nastavenou šablonu. Kapitoly začínají na liché stránce.
\item Technickou zprávu mám v deskách z knihařství (min. 1 výtisk, při utajení oba).
\item Za titulním listem práce je zadání (tzn. mám jej stažené z IS a vložené do šablony).
\item V IS jsou abstrakty a klíčová slova.
  \begin{itemize}
    \item V abstraktu a klíčových slovech v IS nejsou zkopírované vlnky pro nezlomitelné mezery.
  \end{itemize}      
\item V IS je PDF práce (s klikatelnými odkazy).
\item Oba výtisky práce jsou podepsané.
\item V jednom (při utajení obou) výtisku práce je paměťové médium, na kterém je fixkou napsaný login (fixku na CD lze zapůjčit v knihovně, na Studijním oddělení nebo až při odevzdání).
\end{itemize}


\section{\LaTeX pro začátečníky}
\label{latex}

V této kapitole jsou uvedeny některé často využívané balíčky a příkazy pro \LaTeX{}, které mohou být při tvorbě práce potřeba.

\subsection*{Užitečné balíčky}

Studenti při sazbě textu často řeší stejné problémy. Některé z nich lze vyřešit následujícími balíčky pro \LaTeX:

\begin{itemize}
  \item \verb|amsmath| -- rozšířené možnosti sazby rovnic,
  \item \verb|float, afterpage, placeins| -- úprava umístění obrázků/tabulek (specifikátor \texttt{H}),
  \item \verb|fancyvrb, alltt| -- úpravy vlastností prostředí Verbatim, 
  \item \verb|makecell| -- rozšíření možností tabulek,
  \item \verb|pdflscape, rotating| -- natočení stránky o 90 stupňů (pro obrázek či tabulku),
  \item \verb|hyphenat| -- úpravy dělení slov,
  \item \verb|picture, epic, eepic| -- přímé kreslení obrázků.
\end{itemize}

Některé balíčky jsou využity přímo v šabloně (v dolní části souboru \texttt{fitthesis.cls}). Nahlédnutí do jejich dokumentace může být rovněž velmi užitečné.

Sloupec tabulky zarovnaný vlevo s pevnou šířkou je v šabloně definovaný \uv{L} (používá se jako \uv{p}).

Pro odkazování v rámci textu použijte příkaz \verb|\ref{navesti}|. Podle umístění návěští se bude jednat o~číslo kapitoly, podkapitoly, obrázku, tabulky nebo podobného číslovaného prvku). Pokud chcete odkázat stránku práce, použijte příkaz \verb|pageref{navesti}|. Pro citaci literárního odkazu \verb|\cite{identifikator}|. Pro odkazy na rovnice lze použít příkaz \verb|\eqref{navesti}|.

Znak \,--\, (pomlčka) se V \LaTeX u vkládá jako dvě mínus za sebou: -{}-.

\subsection*{Často využívané příkazy pro \LaTeX{}}
\label{sec:Fragments}

Doporučuji nahlédnout do zdrojového textu této podkapitoly a podívat se, jak jsou následující ukázky vysázeny. Ve zdrojovém textu jsou i pomocné komentáře.

% Sloupec zarovnaný vlevo s pevnou šířkou je v šabloně definovaný "L" (používá se jako p)

Příklad tabulky:
\begin{table}[H]
	\vskip6pt
	\caption{Tabulka hodnocení} 
    \vskip6pt
	\centering
	\begin{tabular}{llr}
		\toprule
		\multicolumn{2}{c}{Jméno} \\
		\cmidrule(r){1-2}
		Jméno & Příjmení & Hodnocení \\
		\midrule
		Jan & Novák & $7.5$ \\
		Petr & Novák & $2$ \\
		\bottomrule
	\end{tabular}
	\label{tab:ExampleTable}
\end{table}

% Ohraničení lze upravit dle potřeby:
% http://latex-community.org/forum/viewtopic.php?f=45&t=24323
% http://tex.stackexchange.com/questions/58163/problem-with-multirow-and-table-cell-borders
% http://tex.stackexchange.com/questions/79369/formatting-table-border-and-text-alignment-in-latex-table

\noindent Příklad rovnice:
\begin{equation}
	\cos^3 \theta =\frac{1}{4}\cos\theta+\frac{3}{4}\cos 3\theta
	\label{eq:rovnice2}
\end{equation}
a dvou horizontálně zarovnaných rovnic: % znak & řídí zarovnání
\begin{align} 
    \label{eq:soustava}
	3x &= 6y + 12 \\
	x &= 2y + 4 
\end{align}

Pokud je třeba rovnici citovat v textu, lze použít příkaz \verb|\eqref|. Například na rovnici výše lze odkázat~\eqref{eq:rovnice2}. Pokud chcete srovnat číslo rovnic u soustavy, lze použít prostředí \texttt{split}:
\begin{equation} \label{eq:soustavaSrovnana}
\begin{split}
	3x &= 6y + 12 \\
	x &= 2y + 4
\end{split}
\end{equation}

Matematické symboly ($\alpha$) a výrazy lze umístit i do textu $\cos\pi=-1$ a mohou být i~v~poznámce pod čarou%
\footnote{Vzorec v poznámce pod čarou: $\cos\pi=-1$}.

Obrázek~\ref{sirokyObrazek} ukazuje široký obrázek složený z více menších obrázků. Klasický rastrový obrázek se vkládá tak, jak je vidět na obrázku \ref{keepCalm}.

% Využití \begin{figure*} způsobí, že obrázek zabere celou šířku stránky. Takový obrázek dříve mohl být pouze na začátku stránky, případně na konci s využitím balíčku dblfloatfix (případné [h] se ignorovalo a [H] obrázek odstraní). Nové verze LaTeXu už umí i [h].
\begin{figure*}[h]\centering
  \centering
  \includegraphics[width=\linewidth,height=1.7in]{obrazky/uhk.jpg}\\[1pt]
  \includegraphics[width=0.24\linewidth]{obrazky/uhk.jpg}\hfill
  \includegraphics[width=0.24\linewidth]{obrazky/uhk.jpg}\hfill
  \includegraphics[width=0.24\linewidth]{obrazky/uhk.jpg}\hfill
  \includegraphics[width=0.24\linewidth]{obrazky/uhk.jpg}
  \caption{\textbf{Široký obrázek.} Obrázek může být složen z více menších obrázků. Chcete-li se na tyto dílčí obrázky odkazovat z textu, využijte balíček \texttt{subcaption}.}
  \label{sirokyObrazek}
\end{figure*}

% Odkomentujte pro přepnutí na formát A3 na šířku
% \eject \pdfpagewidth=420mm

\begin{figure}[hbt]
	\centering
	\includegraphics[width=0.3\textwidth]{obrazky/uhk.jpg}
	\caption{Dobrý text je špatným textem, který byl několikrát přepsán. Nebojte se prostě něčím začít.}
	\label{keepCalm}
\end{figure}

Někdy je potřeba do příloh umístit diagram, který se nevejde na stránku formátu A4. Pak je možné vložit jednu stránku formátu A3 a do práce ji poskládat (tzv. skládání do~Z, kdy se vytvoří dva sklady -- lícem dolů a lícem nahoru, angl. Engineering fold -- existuje i~anglický pojem Z-fold, ale při tom by byl problém s vazbou). Přepnutí se provádí následovně: \texttt{\textbackslash{}eject \textbackslash{}pdfpagewidth=420mm} (pro přepnutí zpět pak 210mm).

Další často využívané příkazy naleznete ve zdrojovém textu ukázkového obsahu této šablony.

% Odkomentujte pro přepnutí zpět na A4
% \eject \pdfpagewidth=210mm



  \includepdf[pages=-,offset=0.6cm -1.7cm]{zadani.pdf}
  
\end{document}
