\documentclass[czech]{fim-uhk-thesis}
%\documentclass[english]{fim-uhk-thesis} % without assignment - for the work start to avoid compilation problem
% * Je-li práce psaná v anglickém jazyce, je zapotřebí u třídy použít 
%   parametr english následovně:
%   If thesis is written in English, it is necessary to use 
%   parameter english as follows:
%      \documentclass[english]{fim-uhk-thesis}
% * Je-li práce psaná ve slovenském jazyce, je zapotřebí u třídy použít 
%   parametr slovak následovně:
%   If the work is written in the Slovak language, it is necessary 
%   to use parameter slovak as follows:
%      \documentclass[slovak]{fim-uhk-thesis}
% * Je-li práce psaná v anglickém jazyce se slovenským abstraktem apod., 
%   je zapotřebí u třídy použít parametry english a enslovak následovně:
%   If the work is written in English with the Slovak abstract, etc., 
%   it is necessary to use parameters english and enslovak as follows:
%      \documentclass[english,enslovak]{fim-uhk-thesis}

% Základní balíčky jsou dole v souboru šablony fim-uhk-thesis.cls
% Basic packages are at the bottom of template file fim-uhk-thesis.cls
% zde můžeme vložit vlastní balíčky / you can place own packages here



%---rm---------------
\renewcommand{\rmdefault}{lmr}%zavede Latin Modern Roman jako rm / set Latin Modern Roman as rm
%---sf---------------
\renewcommand{\sfdefault}{qhv}%zavede TeX Gyre Heros jako sf
%---tt------------
\renewcommand{\ttdefault}{lmtt}% zavede Latin Modern tt jako tt

% vypne funkci šablony, která automaticky nahrazuje uvozovky,
% aby nebyly prováděny nevhodné náhrady v popisech API apod.
% disables function of the template which replaces quotation marks
% to avoid unnecessary replacements in the API descriptions etc.
\csdoublequotesoff


\usepackage{url}\def\UrlBreaks{\do\/\do-}
\usepackage{url}
\usepackage{longtable}
\usepackage{tabularx}

\usepackage{xltabular}

\usepackage{tabularray}
\usepackage{xcolor}
\usepackage{pdflscape}
\usepackage{fancyhdr} 

\fancypagestyle{mylandscape}{
\fancyhf{} %Clears the header/footer
\fancyfoot{% Footer
\makebox[\textwidth][r]{% Right
  \rlap{\hspace{.75cm}% Push out of margin by \footskip
    \smash{% Remove vertical height
      \raisebox{4.87in}{% Raise vertically
        \rotatebox{90}{\thepage}}}}}}% Rotate counter-clockwise
\renewcommand{\headrulewidth}{0pt}% No header rule
\renewcommand{\footrulewidth}{0pt}% No footer rule
}




% =======================================================================
% balíček "hyperref" vytváří klikací odkazy v pdf, pokud tedy použijeme pdflatex
% problém je, že balíček hyperref musí být uveden jako poslední, takže nemůže
% být v šabloně
% "hyperref" package create clickable links in pdf if you are using pdflatex.
% Problem is that this package have to be introduced as the last one so it 
% can not be placed in the template file.

  \usepackage{color}
  \usepackage[unicode,colorlinks,hyperindex,plainpages=false,urlcolor=black,linkcolor=black,citecolor=black]{hyperref}
  \definecolor{links}{rgb}{0,0,0}
  \definecolor{anchors}{rgb}{0,0,0}
  \def\AnchorColor{anchors}
  \def\LinkColor{links}
  \def\pdfBorderAttrs{/Border [0 0 0] } % bez okrajů kolem odkazů / without margins around links
  \pdfcompresslevel=9

% Řešení problému, kdy klikací odkazy na obrázky vedou za obrázek
% This solves the problems with links which leads after the picture
\usepackage[all]{hypcap}

\projectinfo{
  %Prace / Thesis
  project={BP},            %typ práce BP/SP/DP/DR  / thesis type (SP = term project)
  year={2020},             % rok odevzdání / year of submission
  date=\today,             % datum odevzdání / submission date
  %Nazev prace / thesis title
  title.cs={Název práce},  % název práce v češtině či slovenštině (dle zadání) / thesis title in czech language (according to assignment)
  title.en={Thesis title}, % název práce v angličtině / thesis title in english
  %title.length={14.5cm}, % nastavení délky bloku s titulkem pro úpravu zalomení řádku (lze definovat zde nebo níže) / setting the length of a block with a thesis title for adjusting a line break (can be defined here or below)
  %sectitle.length={14.5cm}, % nastavení délky bloku s druhým titulkem pro úpravu zalomení řádku (lze definovat zde nebo níže) / setting the length of a block with a second thesis title for adjusting a line break (can be defined here or below)
  %Autor / Author
  author.name={Jméno},   % jméno autora / author name
  author.surname={Příjmení},   % příjmení autora / author surname 
  author.field={Aplikovaná Informatika},   % obor
  %author.title.p={Bc.}, % titul před jménem (nepovinné) / title before the name (optional)
  %author.title.a={Ph.D.}, % titul za jménem (nepovinné) / title after the name (optional)
  %Ustav / Department
  department={DEF}, % doplňte příslušnou zkratku dle ústavu na zadání: DEF / fill in appropriate abbreviation of the department according to assignment: DEF
  % Školitel / supervisor
  supervisor.name={Jméno},   % jméno školitele / supervisor name 
  supervisor.surname={Příjmení},   % příjmení školitele / supervisor surname
  supervisor.title.p={prof. RNDr.},   %titul před jménem (nepovinné) / title before the name (optional)
  supervisor.title.a={Ph.D.},    %titul za jménem (nepovinné) / title after the name (optional)
  % Klíčová slova / keywords
  keywords.cs={Sem budou zapsána jednotlivá klíčová slova v českém (slovenském) jazyce, oddělená čárkami.}, % klíčová slova v českém či slovenském jazyce / keywords in czech or slovak language
  keywords.en={Sem budou zapsána jednotlivá klíčová slova v anglickém jazyce, oddělená čárkami.}, % klíčová slova v anglickém jazyce / keywords in english
  %keywords.en={Here, individual keywords separated by commas will be written in English.},
  % Anotace
  annotation.cs={Text anotace – shrnutí cíle, významu práce a výsledky v ní dosažené. Délka minimálně 100 a maximálně 200 slov},
  annotation.en={Anotace v anglickém jazyce. Délka minimálně 100 a maximálně 200 slov.},
  % Prohlášení (u anglicky psané práce anglicky, u slovensky psané práce slovensky) / Declaration (for thesis in english should be in english)
  declaration={Prohlašuji, že jsem bakalářskou/diplomovou práci zpracoval/zpracovala samostatně a s použitím uvedené literatury},
  %declaration={I hereby declare that this Bachelor's thesis was prepared as an original work by the author under the supervision of Mr. X
% The supplementary information was provided by Mr. Y
% I have listed all the literary sources, publications and other sources, which were used during the preparation of this thesis.},
  % Poděkování (nepovinné, nejlépe v jazyce práce) / Acknowledgement (optional, ideally in the language of the thesis)
  acknowledgment={V této sekci je možno uvést poděkování vedoucímu práce a těm, kteří poskytli odbornou pomoc
(externí zadavatel, konzultant apod.).},
  %acknowledgment={Here it is possible to express thanks to the supervisor and to the people which provided professional help
%(external submitter, consultant, etc.).},
  faculty={FIM}, % TODO FMI/DEF
%   faculty.cs={Fakulta informatiky a managementu}, % Fakulta v češtině - pro využití této položky výše zvolte fakultu DEF / Faculty in Czech - for use of this entry select DEF above
%   faculty.en={Faculty of Informatics and Management}, % Fakulta v angličtině - pro využití této položky výše zvolte fakultu DEF / Faculty in English - for use of this entry select DEF above
  department={KIT} % TODO dopsat seznam
%   department.cs={KATEDRA INFORMAČNÍCH TECHNOLOGIÍ}, % Ústav v češtině - pro využití této položky výše zvolte ústav DEF nebo jej zakomentujte / Department in Czech - for use of this entry select DEF above or comment it out
%   department.en={DEPARTMENT OF INFORMATION TECHNOLOGY} % Ústav v angličtině - pro využití této položky výše zvolte ústav DEF nebo jej zakomentujte / Department in English - for use of this entry select DEF above or comment it out
}

% řeší první/poslední řádek odstavce na předchozí/následující stránce
% solves first/last row of the paragraph on the previous/next page
\clubpenalty=10000
\widowpenalty=10000

\include{text-03-zkratky}
% začátek dokumentu
\begin{document}


  % Vysazeni titulnich stran / Typesetting of the title pages
  % ----------------------------------------------
  \maketitle
  % Obsah
  % ----------------------------------------------
  \setlength{\parskip}{0pt}

  \thispagestyle{empty}
  {\hypersetup{hidelinks}\tableofcontents}
  
  % Seznam obrazku a tabulek (pokud prace obsahuje velke mnozstvi obrazku, tak se to hodi)
  % List of figures and list of tables (if the thesis contains a lot of pictures, it is good)
  
  \ifczech
    \renewcommand\listfigurename{Seznam obrázků}
  \fi
  \ifslovak
    \renewcommand\listfigurename{Zoznam obrázkov}
  \fi
   {\hypersetup{hidelinks}\listoffigures}
   \thispagestyle{empty}
  
  \ifczech
    \renewcommand\listtablename{Seznam tabulek}
  \fi
  \ifslovak
    \renewcommand\listtablename{Zoznam tabuliek}
  \fi
  
  {\hypersetup{hidelinks}\listoftables}
  \thispagestyle{empty}

  % vynechani stranky v oboustrannem rezimu
  % Skip the page in the two-sided mode
  \iftwoside
    \cleardoublepage
  \fi

  % Text prace / Thesis text
  % ----------------------------------------------
  \setcounter{page}{0}
  % ################################
\section{Úvod}

\todo{Zde vysvětlit problémovou situaci a otázky, které se budou v bakalářské/diplomové práci řešit.}

Citace \cite{abadi_tensorflow:_2016, lacey_deep_2016}.


% ################################
\section{Cíl práce}

\todo{Smysl a účel, výzkumné otázky.}


% ################################
\section{Metodika zpracování}

\todo{Cíle, hypotézy/ výzkumné otázky, způsob hledání odpovědí na výzkumné otázky včetně metodiky vlastního výzkumu/šetření, literární rešerše.}


% ################################
\section{Vlastní text práce}

\todo{TODO}

Vlastní řešení dokládá student zpravidla v několika kapitolách. Podle charakteru práce musí student uvážit, zda informace netextové povahy (data, tabulky, obrázky atd.) bude uvádět přímo v textu, nebo je zařadí až za celou práci ve formě příloh, či bude kombinovat oba způsoby. 

Více podrobností viz Metodické pokyny pro vypracování bakalářských a diplomových prací (zveřejňované formou výnosů děkana) a v kurzu MES – Metodologický seminář. 

\subsection{Podkapitola}

Vlastní text práce.

\begin{landscape}
\pagestyle{mylandscape} %Call our predefined page type

\begin{center}
{\renewcommand{\arraystretch}{1.5}%
\begin{xltabular}{1.0 \linewidth}{ % linewidth is for landscape but for center is textwidth 
 | p { 6 em } | X % 6 em je pevná šířka sloupce zde prvního
 | > { \centering\arraybackslash } X 
 | > { \centering\arraybackslash } X 
 | > { \centering\arraybackslash } X 
 | > { \centering\arraybackslash } X
 | > { \centering\arraybackslash } X
 | > { \centering\arraybackslash } X 
 | > { \centering\arraybackslash } X 
 | > { \centering\arraybackslash } X 
 | > { \centering\arraybackslash } X 
 | > { \centering\arraybackslash } X | }
\caption{Přehledová tabulka.} \label{tab:long} \\

\hline
Identifikace & Zařazení & Využití & Verifikace, validace & Zaměření & Počet agentů, rozsah&Zdrojový kód&Využití Pythonu&ODD protokol&Reálný čas&Zdroj dat\\ \hline
\endfirsthead


\multicolumn{11}{c}%

{\tablename\ \thetable{} -- pokračování z předchozí stránky} \\
\hline
Identifikace & Zařazení& Využití& Verifikace, validace & Zaměření& Počet agentů, Rozsah&Zdrojový kód&Využití Pythonu&ODD protokol&Reálný čas& Zdroj dat \\ \hline 
\endhead

\hline \multicolumn{11}{|c|}{{Pokračování na následující stránce}} \\ \hline
\endfoot

\hline
\endlastfoot
Belotti M. C. T. D. et al., 2022 
&evakuace davu
& výzkum a tvorba modelů
&validace
&indoor, outdoor
&54 agentů, koridor 3x38 metrů
&ANO
&základní kód
&NE
&NE 
&Využívá předchozí studie
\\ \hline
Couasnon et al., 2019 

&loď, evakuace
& lodní inženýři, výzkum
& NE
& indoor
& loď - 5. paluba, 180 m dlouhá, 838 agentů
& NE
& základní kód
&NE
&ANO
&Agenti a prostředí generovány Mesa
\\ \hline
Crooks Andrew et al., 2017
 
&evakuace po výbuchu
&IZS, výzkum
&ANO
&outdoor
& oblast 262 x 234 km, 22 795 866 osob, 225 906 km silnic
&uveden nefunkční odkaz
&zpracová-ní a mapování dat
&NE
&NE
& Pomocí smíšené metody syntézy použity údaje ze sčítání lidu
\\ \hline
\end{xltabular}}
\end{center}
\end{landscape}


\subsubsection{Podřazená kapitola}
Text s odkazem na obrázek \ref{figure:uhk}.


\begin{figure}[hbt!]
 	\begin{center}
    	\includegraphics[width=\textwidth]{obrazky/uhk.jpg}
 	\end{center}
 	\caption{Ukázkový obrázek}
	\label{figure:uhk}
\end{figure} 


% ################################
\section{Souhrn výsledků}

\todo{Souhrn vlastních výsledků získaných v průběhu řešení problému.}

\noindent\todo{Zkratka} \Ac{DNN}


% ################################
\section{Závěry a doporučení}

\todo{Kritická diskuze nad výsledky, ke kterým autor dospěl (soulad výsledků  literaturou či předpoklady; výsledky a okolnosti, které zvláště ovlivnily předkládanou práci atd.). Je vhodné naznačit i případné další (popř. alternativní) možnosti zkoumání dané problematiky a otevřené problémy pro další studium. }


  \iftwoside
    \cleardoublepage
  \fi

  % Pouzita literatura / Bibliography
  % ----------------------------------------------

\ifslovak
  \makeatletter
  \def\@openbib@code{\addcontentsline{toc}{section}{Literatúra}}
\else
  \ifczech
    \makeatletter
    \def\@openbib@code{\addcontentsline{toc}{section}{Literatura}}
  \else 
    \makeatletter
    \def\@openbib@code{\addcontentsline{toc}{section}{Bibliography}}
  \fi
\fi
  \makeatother
  \begin{flushleft}
  \label{references}

  \bibliographystyle{iso690}
  \bibliography{text-02-literatura}
  \end{flushleft}
  
  \newpage

  % vynechani stranky v oboustrannem rezimu
  % Skip the page in the two-sided mode
  \iftwoside
    \cleardoublepage
  \fi
  
  \ifslovak
   \addcontentsline{toc}{section}{Zoznam zkratek}
  \else
    \ifczech
      \addcontentsline{toc}{section}{Seznam zkratek}
    \else
      \addcontentsline{toc}{section}{List of Abbreviations}
    \fi
  \fi
  \startcontents[section]
  \label{abb}
  \setlength{\parskip}{0pt} 
  % seznam příloh / list of appendices
  % \printcontents[section]{l}{0}{\setcounter{tocdepth}{2}}
  
  \printacronyms[name=Seznam zkratek,heading=section*]
  
  \newpage
  % vynechani stranky v oboustrannem rezimu
  \iftwoside
    \cleardoublepage
  \fi

  % Prilohy / Appendices
  % ---------------------------------------------
  \appendix

    \ifczech
    \renewcommand{\appendixpagename}{Přílohy}
    \renewcommand{\appendixtocname}{Přílohy}
    \renewcommand{\appendixname}{Příloha}
  \fi
  \ifslovak
    \renewcommand{\appendixpagename}{Prílohy}
    \renewcommand{\appendixtocname}{Prílohy}
    \renewcommand{\appendixname}{Príloha}
  \fi
 \appendixpage
  
  % Přílohy / Appendices
  \input{text-04-prilohy}

  \includepdf[pages=-,offset=0.6cm -1.7cm]{zadani.pdf}
  
\end{document}
